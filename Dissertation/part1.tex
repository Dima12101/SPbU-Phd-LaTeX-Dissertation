\chapter{АНАЛИЗ ПРОБЛЕМЫ И ОБОСНОВАНИЕ ПОДХОДА}\label{ch:ch1}

\section{Проблемная область: управление ресурсами в распределённых IoT-системах}\label{sec:ch1/sec1}

\subsection{Характеристики современных IoT и edge-систем}\label{sec:ch1/sec1/subsec1}

Развитие Интернета вещей (IoT) привело к появлению массово распределённых сетей, состоящих из миллионов гетерогенных устройств с ограниченными вычислительными возможностями~\cite{ref22}. Согласно последним исследованиям, глобальное количество подключённых IoT-устройств превысило 15 миллиардов, и эта цифра продолжает расти экспоненциально~\cite{ref40}. Одновременно с этим растёт объём данных, генерируемых этими устройствами, создавая новые вызовы для систем обработки и управления.

Централизованная парадигма облачных вычислений, исторически применяемая для обработки больших объёмов данных, становится неприемлемой в условиях IoT по нескольким причинам~\cite{ref34}:

\begin{enumerate}
	\item \textbf{Высокая задержка (latency).} Передача данных на удалённый облачный сервер и получение результата требует времени, несовместимого с требованиями критических приложений (медицина, автономный транспорт, промышленная автоматизация);
	\item \textbf{Ограничение пропускной способности сети.} Передача всех данных в облако требует огромной пропускной способности каналов связи, что часто невозможно в географически распределённых и мобильных сетях;
	\item \textbf{Проблемы конфиденциальности и безопасности.} Централизованное хранение чувствительных данных создаёт значительные риски;
	\item \textbf{Энергоэффективность.} Для мобильных и портативных устройств передача больших объёмов данных на дальние расстояния является энергетически расточительной.
\end{enumerate}

\begin{figure}[h]
	\centering
	\includegraphics[width=10cm, height=6cm]{arch-cloud-fog-edge}
	\caption{Архитектура Cloud-Fog-Edge}
	\label{img:arch-cloud-fog-edge}
\end{figure}

Эти вызовы привели к развитию \textbf{парадигмы вычислений на периферии сети (edge computing)}~\cite{ref46}, которая предполагает размещение вычислительных ресурсов близко к источникам данных. В архитектуре облачно-периферийных систем (Cloud-Fog-Edge) (см. рис. \ref{img:arch-cloud-fog-edge}) вычисления распределены по нескольким уровням: централизованное облако, промежуточные узлы (fog servers), граничные узлы (edge servers) и конечные устройства IoT. Исследования показывают, что edge computing может сократить задержку на 80\% по сравнению с полностью облачными решениями.

\subsection{Фундаментальная проблема: распределение ограниченных ресурсов}\label{sec:ch1/sec1/subsec2}

Несмотря на размещение вычислительной мощности на граничных узлах, возникает новая, не менее сложная проблема: \textbf{эффективное распределение ограниченных вычислительных ресурсов между конкурирующими агентами (устройствами IoT)}~\cite{ref40}.

Граничные узлы (edge servers), в отличие от центральных облачных центров данных, имеют принципиально ограниченные ресурсы:
\begin{itemize}
	\item процессорной мощности (несколько ядер против сотен в облаке);
	\item объёма оперативной памяти (десятки ГБ против терабайт);
	\item энергетической ёмкости (для мобильных граничных узлов);
	\item пропускной способности каналов связи.
\end{itemize}

В то же время спрос на вычислительные ресурсы от множества устройств IoT может быть произвольным, непредсказуемым и резко меняющимся~\cite{ref36}.

\textbf{Центральная задача}, на решение которой направлено данное исследование: \textit{как справедливо, эффективно и адаптивно распределять ограниченные вычислительные ресурсы граничных узлов между множеством конкурирующих устройств IoT в условиях неполной информации, динамической нагрузки и без централизованного управления?}

\subsection{Децентрализованное управление в распределённых системах}\label{sec:ch1/sec1/subsec3}

Классический подход к управлению ресурсами в централизованных системах предполагает наличие единого администратора (контроллера), который имеет полную информацию о состоянии системы и может принимать глобально оптимальные решения. Однако в распределённых IoT-системах такой администратор либо отсутствует, либо не имеет полной информации \cite{ref35}.

Это создаёт два взаимосвязанных \textbf{вызова}:
\begin{enumerate}
	\item \textit{Проблема стимулирования (incentive problem):} необходимо спроектировать механизм, который заставляет участников действовать честно и в соответствии с общей целью;
	\item \textit{Проблема адаптивности (adaptivity problem):} параметры механизма должны динамически адаптироваться к изменяющимся условиям среды.
\end{enumerate}

\section{Анализ существующих подходов и их ограничения}\label{sec:ch1/sec2}

\subsection{Детерминированные оптимизационные методы}\label{sec:ch1/sec2/subsec1}

Исторически первым подходом к управлению ресурсами в распределённых системах были детерминированные алгоритмы оптимизации \cite{ref29}. Они формулируют задачу распределения как задачу математической оптимизации:

\begin{equation}
	\begin{aligned}
		& \text{minimize} \quad f(x) \\
		& \text{subject to} \quad g_i(x) \leq 0, \quad h_j(x) = 0
	\end{aligned}
\end{equation}

где $x$ — вектор переменных распределения ресурсов, $f$ — целевая функция, а ограничения отражают доступность ресурсов.

Такой подход дает ряд \textbf{преимуществ:}
\begin{itemize}
	\item Гарантированная оптимальность решения (для выпуклых задач);
	\item Хорошо развитая математическая теория и численные методы;
	\item Предсказуемое время выполнения.
\end{itemize}

Но при этом имеет критические \textbf{ограничения:}

\begin{enumerate}
	\item Предположение о полной информации: алгоритмы предполагают, что функции $f$ и $g_i$ известны и неизменны. В реальных системах это редко бывает правдой;
	\item Статичность: параметры оптимизации фиксированы и не адаптируются к изменению условий;
	\item Централизованность: большинство методов требуют глобальной информации для выполнения алгоритма;
	\item Масштабируемость: вычислительная сложность часто растёт полиномиально или экспоненциально с размером системы~\cite{ref35};
	\item Отсутствие учёта стимулов: алгоритмы не рассчитывают на возможность стратегического поведения участников.
\end{enumerate}

\subsection{Аукционные механизмы}\label{sec:ch1/sec2/subsec2}

Аукционные механизмы были предложены как способ справиться с проблемой стимулирования в распределённых системах. Идея основана на экономической теории: если участники конкурируют за ресурсы через процесс торгов, они будут мотивированы раскрывать свои истинные предпочтения~\cite{ref27, ref41}.

\textbf{Механизм VCG} (Vickrey-Clarke-Groves) является наиболее известным представителем этого семейства~\cite{ref69, ref72}. Его ключевое свойство — доминантно-стратегическая совместимость (Dominant Strategy Incentive Compatibility, \textbf{DSIC}) — гарантирует, что для участника всегда выгодно раскрывать свои истинные предпочтения независимо от действий других участников.

Формально, в механизме VCG платёж участнику $i$ определяется как:

\begin{equation}
	p_i(v) = \sum_{j \neq i} v_j(x^*) - \sum_{j \neq i} v_j(x^*_{-i})
\end{equation}

где $x^*$ — социально оптимальное распределение, а $x^*_{-i}$ — оптимальное распределение без участника $i$. Эта формула гарантирует правдивость раскрытия предпочтений.

Тем самым, такой подход имеет следующие \textbf{теоретические достоинства:}
\begin{itemize}
	\item Гарантированная честность (truth-telling is a dominant strategy);
	\item Максимизация социального благосостояния;
	\item Индивидуальная рациональность.
\end{itemize}

Но в чистом виде имеет \textbf{практические ограничения} в контексте IoT:
\begin{enumerate}
	\item \textit{Статичность параметров:} Классический VCG предполагает, что функции полезности $v_i(x)$ известны и неизменны \cite{ref27}. В динамичных IoT-системах эти функции зависят от состояния сети, доступности устройств, характера рабочей нагрузки, которые постоянно меняются. Фиксированный механизм становится неэффективным \cite{ref40};
	\item \textit{Отсутствие адаптивности:} Даже если параметры механизма оптимальны для одного момента времени, они могут быть далеко от оптимальных в следующий момент;
	\item \textit{Отсутствие обучения:} VCG предполагает, что участники раскрывают свои истинные предпочтения. Однако в реальных системах участники могут обучаться на основе предыдущего поведению других участников;
	\item \textit{Вычислительная сложность:} Определение оптимального распределения часто требует решения сложной оптимизационной задачи. В масштабных системах это может быть вычислительно неосуществимо~\cite{ref27};
	\item \textit{Проблема полной информации:} Для гарантии DSIC механизм VCG требует знания всех функций полезности.
\end{enumerate}

\subsection{Методы машинного обучения с подкреплением (RL)}\label{sec:ch1/sec2/subsec3}

В последнее десятилетие методы обучения с подкреплением получили широкое распространение для управления ресурсами в IoT и edge-вычислениях \cite{ref22, ref26, ref30}. Главная идея: вместо предположения о полной информации, система учится из опыта, взаимодействуя с окружением (через действия и награды) (см. рис. \ref{img:arch-rl}).

\begin{figure}[h]
	\centering
	\includegraphics[width=10cm, height=4cm]{arch-rl}
	\caption{Общая схема RL-методов}
	\label{img:arch-rl}
\end{figure}

Таким образом, получаем следующие \textbf{преимущества} RL-методов:
\begin{itemize}
	\item Адаптивность к изменяющимся условиям;
	\item Способность работать в условиях неполной информации;
	\item Автоматическое обучение оптимальных стратегий.
\end{itemize}

Но при этом также имемм критические \textbf{ограничения:}

\begin{enumerate}
	\item \textit{Отсутствие гарантий стимулирования:} RL-методы оптимизируют целевую функцию системы, но не гарантируют, что это совпадает с интересами отдельных агентов. Нет гарантий DSIC \cite{ref22};
	\item \textit{Нестабильность обучения в многоагентной среде:} Когда несколько агентов обучаются одновременно, среда становится нестационарной~\cite{ref22};
	\item \textit{Проблема credit assignment:} В кооперативной многоагентной среде сложно определить, какой агент внёс вклад в результат~\cite{ref42};
	\item \textit{Требование исследования (exploration):} RL-методы требуют значительного времени на исследование;
	\item \textit{Отсутствие теоретических гарантий:} Большинство RL-методов не обеспечивают теоретических гарантий конвергенции~\cite{ref42}.
\end{enumerate}

Многоагентное RL (\textbf{MARL}) прямо ориентировано на решение проблем в системах с несколькими агентами~\cite{ref42, ref50}. Однако стандартные методы MARL часто игнорируют проблему стимулирования и не адаптируют свои решения на основе теории игр.

\subsection{Формализм Dec-POMDP для многоагентных систем}\label{sec:ch1/sec2/subsec4}

Частично наблюдаемые марковские процессы принятия решений (\textbf{POMDP}) (см. рис. \ref{img:pomdp}) представляют собой мощный математический формализм для моделирования систем, где агент имеет ограниченную информацию о состоянии окружения и должен принимать решения на основе локальных наблюдений. 

\begin{figure}[h]
	\centering
	\includegraphics[width=8cm, height=4cm]{pomdp}
	\caption{Концептуальная схема POMDP}
	\label{img:pomdp}
\end{figure}

Для многоагентных систем, децентрализованный POMDP (\textbf{Dec-POMDP}) позволяет моделировать координацию между несколькими независимо действующими агентами в условиях неполной информации при сохранении кооперативного достижения общей цели~\cite{ref68, ref71}.Он используется в последних работах по динамическому спектральному доступу \cite{ref57}, распределению задач в UAV-системах \cite{ref58} и другим приложениям.

Его \textbf{достоинствами} являются:
\begin{itemize}
	\item Точная математическая модель для многоагентных систем;
	\item Формальное определение оптимальности;
	\item Возможность доказывать свойства решений.
\end{itemize}

Но также имеет \textbf{арактические ограничения:}
\begin{enumerate}
	\item \textit{Вычислительная неразрешимость:} Решение Dec-POMDP в общем случае является \textit{NEXP-полной задачей} (double exponential complexity)~\cite{ref60}, что означает, что не существует эффективного алгоритма для общего случая~\cite{ref63};
	\item \textit{Несовместимость с аукционными механизмами:} Классический Dec-POMDP не учитывает экономические стимулы~\cite{ref68};
	\item \textit{Требование полной модели:} Решение Dec-POMDP обычно требует знания функций переходов и наблюдений. В реальных системах эти функции часто неизвестны~\cite{ref71}.
\end{enumerate}

Несмотря на эти ограничения, Dec-POMDP остаётся ценным инструментом для формализации и анализа многоагентных систем, когда он комбинируется с приблизительными методами решения.

\subsection{Пробелы в существующих подходах}\label{sec:ch1/sec2/subsec5}

На основе проведённого анализа можно выделить несколько критических пробелов:
\begin{enumerate}
	\item \textbf{Отсутствие интеграции аукционов и MARL}. Существует явный разрыв между теорией аукционов (обеспечивающей справедливость) и методами машинного обучения (обеспечивающими адаптивность). В литературе практически отсутствуют работы, которые интегрировали бы эти два подхода в единую систему управления ресурсами.
	\item \textbf{Отсутствие динамических (адаптивных) аукционов}. Большинство работ рассматривают аукционы как одноразовый процесс. Однако в динамичных IoT-системах аукцион повторяется многократно, и параметры должны адаптироваться. Работы по адаптивным аукционам остаются редкостью.
	\item \textbf{Ограниченное число приложений к реальным сценариям}. Большинство исследований ограничиваются теоретическими моделями или симуляциями на синтетических данных. Практическое применение к реальным предметным областям (роботизированные склады, системы автономного транспорта) остаётся малоизученным.
	\item \textbf{Отсутствие анализа справедливости в динамичных системах}. Существующие работы по справедливости распределения ресурсов обычно рассматривают статичные сценарии. Анализ долгосрочной справедливости в системах, где параметры обучаются и адаптируются, отсутствует.
\end{enumerate}

\section{Обоснование выбора интегрированного подхода}\label{sec:ch1/sec3}

На основе анализа ограничений существующих методов можно обосновать необходимость \textbf{интегрированного подхода}, объединяющего:
\begin{enumerate}
	\item \textbf{Аукционные механизмы (VCG)} -- для обеспечения справедливости и стимулирования честного поведения;
	\item \textbf{Многоагентное обучение с подкреплением (MARL)} -- для адаптации параметров аукциона к динамическим условиям;
	\item \textbf{Dec-POMDP формализм} -- для строгого математического описания многоагентной системы с неполной информацией.
\end{enumerate}

\textbf{VCG + MARL:} VCG обеспечивает теоретическое основание для справедливого распределения ресурсов; MARL позволяет автоматически обучать параметры функций на основе наблюдаемых результатов; результирующая система сочетает теоретические гарантии с практической адаптивностью.

\textbf{MARL + Dec-POMDP:} Dec-POMDP даёт формальное описание многоагентной системы; современные MARL-алгоритмы (особенно QMIX) эффективно решают Dec-POMDP-задачи с линейной масштабируемостью; парадигма CTDE (Centralized Training with Decentralized Execution) позволяет использовать полную информацию во время обучения, но гарантирует полностью децентрализованное исполнение.

\textbf{VCG + Dec-POMDP:} VCG платежи могут быть интегрированы в функцию вознаграждения Dec-POMDP; это обеспечивает, что агенты, обучаясь максимизировать своё локальное вознаграждение, одновременно способствуют максимизации социального благосустояния.

Далее во второй главе будут представлены детальные описания алгоритмов и архитектура системы, реализующей этот интегрированный подход.
