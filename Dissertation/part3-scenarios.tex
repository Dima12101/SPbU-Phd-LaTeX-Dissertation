\section{Сценарии тестирования}\label{ch:ch3/sec1}

\subsection{Сценарий 1: Базовый сценарий (Baseline)}\label{ch:ch3/sec1/subsec1}

\textbf{Описание}: Базовый сценарий моделирует простую edge-сеть с 4 узлами и 100 мобильными устройствами, генерирующими задачи в течение 2000 временных шагов.

\textbf{Параметры}:
\begin{itemize}
	\item Количество edge-узлов: $n = 4$
	\item Количество мобильных устройств: $m = 100$
	\item Интенсивность прихода задач: $\lambda = 2.5$ (задач на узел в единицу времени)
	\item Длительность эпизода: $T = 2000$ временных шагов
	\item Частота оффлайн-обучения: каждые 100 шагов
	\item Размер Experience Reply Buffer: 10,000 переходов
\end{itemize}

\textbf{Ожидаемые результаты}:
\begin{itemize}
	\item Средняя задержка выполнения: $< 150$ мс
	\item Процент принятых задач: $> 90\%$
	\item Социальное благополучие: растущий тренд
	\item Справедливость платежей: коэффициент Джини платежей $< 0.3$
\end{itemize}

\subsection{Сценарий 2: Высокая нагрузка (High Load)}\label{ch:ch3/sec1/subsec2}

\textbf{Описание}: Тестирование поведения системы при повышенной интенсивности прихода задач.

\textbf{Параметры}:
\begin{itemize}
	\item Количество edge-узлов: $n = 4$
	\item Интенсивность прихода задач: $\lambda = 5.0$ (в 2 раза больше)
	\item Количество мобильных устройств: $m = 150$
	\item Длительность эпизода: $T = 2000$ временных шагов
\end{itemize}

\textbf{Ожидаемые результаты}:
\begin{itemize}
	\item Система должна адаптироваться к перегрузке
	\item Контролируемое увеличение отклонений задач
	\item Балансирование нагрузки между узлами
	\item Стабильность платежей (не скачут резко)
\end{itemize}

\subsection{Сценарий 3: Гетерогенные узлы (Heterogeneous Nodes)}

\textbf{Описание}: Сценарий с узлами разной производительности (облачный центр, промежуточные узлы, мобильные устройства).

\textbf{Параметры}:
\begin{itemize}
	\item Узлы класса A (быстрые): 2 узла, CPU=4, bandwidth=100 Mbps
	\item Узлы класса B (средние): 2 узла, CPU=2, bandwidth=50 Mbps
	\item Узлы класса C (медленные): 2 узла, CPU=1, bandwidth=20 Mbps
	\item Всего узлов: $n = 6$
	\item Мобильные устройства: $m = 120$
\end{itemize}

\textbf{Ожидаемые результаты}:
\begin{itemize}
	\item Распределение: предпочтение быстрых узлов для срочных задач
	\item Справедливость: компенсация за ограниченные ресурсы
	\item Эффективность: использование разнообразных ресурсов
\end{itemize}

\subsection{Сценарий 4: Динамические условия (Dynamic Conditions)}

\textbf{Описание}: Моделирование изменяющихся условий сети (отказы узлов, изменение нагрузки).

\textbf{Параметры}:
\begin{itemize}
	\item Время 0-500: нормальная работа
	\item Время 500-700: один узел выходит из строя
	\item Время 700-1200: восстановление, высокая нагрузка
	\item Время 1200-2000: стабилизация с 5 узлами
	\item Отказоустойчивость: система должна адаптироваться
\end{itemize}

\textbf{Ожидаемые результаты}:
\begin{itemize}
	\item Быстрая адаптация к отказу узла ($< 100$ шагов)
	\item Восстановление производительности
	\item Отсутствие потери справедливости
\end{itemize}
