\chapter{ОПИСАНИЕ АЛГОРИТМОВ}\label{ch:ch2}

\section{Формализация задачи распределения ресурсов}\label{sec:ch2/sec1}

\subsection{Система как граф}\label{sec:ch2/sec1/subsec1}

Систему управления ресурсами можно представить как граф:

\begin{equation}
	G = (V, E), \quad V = N \cup D, \quad N = \{n_1, n_2, \ldots, n_k\}, \quad D = \{d_1, d_2, \ldots, d_m\}
\end{equation}

где:
\begin{itemize}
	\item $N$ -- множество edge-узлов (поставщиков вычислительных ресурсов);
	\item $D$ -- множество IoT-устройств (потребителей ресурсов);
	\item $E$ -- множество рёбер сетевой топологии;
	\item вес каждого ребра $w(d_i, n_j)$ отражает сетевую задержку между устройством и узлом.
\end{itemize}

Такая графовая модель позволяет естественным образом представить иерархическую структуру облачно-граничной системы, где различные edge-узлы могут быть на разных уровнях иерархии.

\subsection{Характеристики ресурсов и требования задач}\label{sec:ch2/sec1/subsec2}

Каждый edge-узел $n_j \in N$ характеризуется вектором доступных ресурсов:

\begin{equation}
	\text{Cap}_j = (\text{CPU}_j, \text{MEM}_j, \text{Bandwidth}_j, \text{Storage}_j)
\end{equation}

и текущей загрузкой:

\begin{equation}
	\text{Load}_j(t) = (\text{CPU}_{\text{avail}}(t), \text{MEM}_{\text{avail}}(t), \ldots)
\end{equation}

которая меняется со временем в зависимости от уже выполняемых задач.

Каждое устройство $d_i \in D$ генерирует вычислительные задачи из множества $T_i(t)$ в момент времени $t$, где каждая задача $\tau$ характеризуется:
\begin{itemize}
	\item \textbf{Требования к ресурсам:} $(\text{CPU}(\tau), \text{MEM}(\tau), \text{Bandwidth}(\tau))$;
	\item \textbf{Объём данных:} $\text{DataSize}(\tau)$;
	\item \textbf{Временной дедлайн:} $\text{Deadline}(\tau)$;
	\item \textbf{Приоритет:} $\text{Priority}(\tau) \in [0, 1]$;
	\item \textbf{Тип задачи:} $\text{Type}(\tau)$ (интерактивная, batch, критическая и т.д.).
\end{itemize}
