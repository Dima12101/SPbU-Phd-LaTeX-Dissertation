\chapter*{ВВЕДЕНИЕ}                         % Заголовок
\addcontentsline{toc}{chapter}{ВВЕДЕНИЕ}    % Добавляем его в оглавление

\newcommand{\actuality}{} 
\newcommand{\progress}{}
\newcommand{\aim}{{\textbf\aimTXT}}
\newcommand{\tasks}{\textbf{\tasksTXT}}
\newcommand{\novelty}{\textbf{\noveltyTXT}}
\newcommand{\influence}{\textbf{\influenceTXT}}
\newcommand{\methods}{\textbf{\methodsTXT}}
\newcommand{\defpositions}{\textbf{\defpositionsTXT}}
\newcommand{\reliability}{\textbf{\reliabilityTXT}}
\newcommand{\probation}{\textbf{\probationTXT}}
\newcommand{\contribution}{\textbf{\contributionTXT}}
\newcommand{\publications}{\textbf{\publicationsTXT}}

Развитие Интернета вещей (IoT), облачных вычислений и мобильных сетей привело к фундаментальному переосмыслению архитектуры распределенных систем обработки информации. Традиционная облачная парадигма, основанная на централизованной обработке данных в удаленных дата-центрах, показала существенные ограничения при работе с приложениями, требующими сверхнизких задержек и локальной обработки больших объемов данных. В ответ на эти вызовы сформировалась парадигма многоуровневого вычисления на периферии сети (\textbf{edge computing}), которая распределяет вычислительные ресурсы и функции обработки данных между устройствами на периферии сети, промежуточными узлами (\textbf{fog computing}) и облачными центрами обработки.

В контексте производственных систем и Индустрии 4.0, в частности при организации роботизированного производства и автоматизированных складов, многоуровневые edge-инфраструктуры обеспечивают необходимые возможности для реализации интеллектуального управления. Каждый робот или автономное мобильное устройство (Autonomous Mobile Robot, AMR) может выступать как вычислительный агент, способный принимать локальные решения, взаимодействовать с соседними агентами и адаптироваться к динамически изменяющимся условиям производственной среды.

Несмотря на потенциал edge computing, его применение в гетерогенных производственных средах сталкивается с рядом фундаментальных проблем:
\begin{enumerate}
    \item Динамичность топологии сети: Физическое расположение устройств, доступность соединений и пропускная способность каналов связи непрерывно изменяются. В сценариях с мобильными роботами и автономными системами изменение топологии происходит в реальном времени, что требует адаптивных механизмов, чувствительных к текущему состоянию сети.
    \item Вариативность сетевых задержек: Задержки передачи данных (network latency) между компонентами системы подвержены значительным колебаниям, зависящим от перегруженности канала, наличия интерференции и физических препятствий. Это приводит к непредсказуемому влиянию на время выполнения распределенных вычислений.
    \item Гетерогенность вычислительных ресурсов: Edge-инфраструктуры часто состоят из устройств с различными вычислительными характеристиками -- от  маломощных IoT-сенсоров до мощных пограничных серверов. Эта гетерогенность усложняет проблему распределения задач и выделения вычислительных ресурсов.
    \item Гетерогенность задач: Приложения, поступающие в систему, имеют разнообразные требования к вычислительным ресурсам (CPU, память, хранилище), различные временные ограничения и т.д.
    \item Необходимость локального принятия решений: Из-за ограничений пропускной способности и требований к минимизации задержек централизованное управление становится неприемлемым. Требуется подход, позволяющий каждому агенту принимать решения на основе локальной информации, одновременно обеспечивая глобальную согласованность и оптимальность.
\end{enumerate}

В этих условиях требуется разработка адаптивных моделей и методов управления ИТ‑ресурсами, что имеет ключевое значение для цифровизации промышленности 4.0 и реализации концепции Industry 5.0.


\textbf{Актуальность темы.} Классические подходы к управлению ресурсами, основанные на статических политиках, неэффективны в динамичной среде. Существующие решения для управления ресурсами в облачных вычислениях предполагают относительную стабильность топологии и предсказуемость характеристик канала. Одновременно увеличение масштабности систем и требования к локальному принятию решений делают централизованное управление неэффективным, поскольку оно создает узкие места, единые точки отказа и не способно оперативно адаптироваться к динамическим изменениям в сети.

Одновременно с этим возрастает требование справедливого и эффективного распределения вычислительных ресурсов между конкурирующими участниками, каждый из которых преследует собственные интересы. Классические оптимизационные методы не учитывают экономические стимулы, необходимые для мотивации устройств к активному участию в совместном использовании ресурсов. Решение этой проблемы требует интеграции теории игр, механизмов аукциона и современных методов искусственного интеллекта.

В этом контексте предложенный подход, объединяющий \textbf{аукционные механизмы} (в частности, механизм Викри-Кларка-Гровса) с  \textbf{многоагентным обучением с подкреплением} (Multi-Agent Reinforcement Learning, MARL) и частично наблюдаемыми марковскими процессами принятия решений (Decentralized Partially Observable Markov Decision Process, Dec-POMDP), представляет собой актуальное и перспективное направление исследований.

\textbf{Целью} настоящий диссертации является разработка интегрированного подхода к управлению распределёнными вычислительными ресурсами в многоагентных edge-инфраструктурах на основе сочетания аукционных механизмов (VCG), многоагентного обучения с подкреплением (MARL) и децентрализованных POMDP (Dec-POMDP), обеспечивающий справедливое, эффективное и адаптивное распределение ресурсов в условиях неполной информации, непредсказуемой топологии сети, вариативных сетевых задержек и гетерогенности как вычислительных ресурсов, так и обрабатываемых задач.

Для достижения поставленной цели необходимо решить следующие \textbf{задачи}.
\begin{enumerate}
	\item Анализ и синтез архитектуры системы. Разработать концептуальную архитектуру, объединяющую три ключевых компонента: аукционный слой (для формирования рынка ресурсов), слой MARL (для адаптивной настройки параметров аукциона) и смешивающую сеть (Mixing Network) для централизованного обучения с децентрализованным исполнением.
	\item Формализация моделей. Предложить математические формализмы для: 1) Функций полезности и стоимости участников аукциона; 2) Механизма VCG в контексте распределения вычислительных ресурсов; 3) Переформулировки задачи как Dec-POMDP с учётом частичной наблюдаемости состояния.
	\item Разработка и верификация алгоритмов. Реализовать два основных алгоритма: 1) MA-VCG (Multi-Agent VCG): детерминированный аукционный механизм для однократного распределения ресурсов; 2) Dec-POMDP-MARL: адаптивный алгоритм обучения на основе QMIX, позволяющий агентам корректировать параметры аукциона на основе наблюдаемой обратной связи.
	\item Экспериментальная оценка и валидация. Провести комплексную оценку алгоритмов на симуляционных и реальных данных в различных сценариях: стабильная нагрузка, динамические пики, отказы узлов, гетерогенные ресурсы.
	\item Применение к практическому сценарию. Разработать и протестировать применение предложенного подхода к управлению ресурсами в роботизированных складах, где множество автономных мобильных роботов конкурируют за доступ к вычислительным ресурсам (обработка видеопотоков, координация движения, планирование маршрутов).
	\item Формальный анализ свойств. Доказать или привести эмпирические свидетельства наличия следующих свойств разработанных механизмов:
	\begin{itemize}
		\item Доминантно-стратегическая совместимость (DSIC) для механизма VCG;
		\item Конвергенция обучения в многоагентной среде;
		\item Справедливость распределения (индекс Джайна);
		\item Масштабируемость (линейная зависимость времени выполнения от числа агентов).
	\end{itemize}
\end{enumerate}

\textbf{Научная новизна} диссертации состоит в следующем:
\begin{enumerate}
	\item Первое интегрированное решение, объединяющее VCG-аукционы с многоагентным обучением для адаптивного управления вычислительными ресурсами в распределённых edge-инфраструктурах.
	\item Применение Dec-POMDP формализма к задаче распределения ресурсов с использованием алгоритма QMIX, обеспечивающего линейную масштабируемость и децентрализованное исполнение.
	\item Архитектура с централизованным обучением и децентрализованным исполнением (Centralized Training with Decentralized Execution, CTDE), позволяющая гибко адаптировать параметры аукциона к изменяющимся условиям.
	\item Комплексная верификация и валидация предложенных алгоритмов как на синтетических, так и на реалистичных сценариях, включая анализ чувствительности к параметрам среды (задержки, дедлайны, пиковые нагрузки).
	\item Приложение к реальной предметной области (роботизированные склады), демонстрирующее практическую применимость подхода.
\end{enumerate}

\textbf{Теоретическая значимость}. 
\begin{itemize}
	\item Углубление понимания взаимодействия между теорией игр (аукционные механизмы) и методами машинного обучения в контексте децентрализованного управления;
	\item Развитие методов анализа многоагентных систем, действующих в условиях неполной информации и конкурентного взаимодействия;
	\item Доказательство возможности достижения совместимости по стимулам (DSIC) в адаптивных многоагентных системах с обучением;
	\item Расширение теоретических основ применения Dec-POMDP к масштабным распределённым системам.
\end{itemize}

\textbf{Практическая значимость}. 
\begin{itemize}
	\item Разработанные алгоритмы могут быть применены к управлению ресурсами в реальных IoT-системах, включая облачно-граничные (cloud-fog) архитектуры;
	\item Подход применим к таким предметным областям, как:
	\begin{itemize}
		\item Роботизированная логистика и автоматизированные склады;
		\item Управление мобильными сенсорными сетями;
		\item Координация распределённых вычислительных ресурсов в 5G/6G сетях;
		\item Системы автономного транспорта (AVs) с локальной обработкой данных;
	\end{itemize}
	\item Решение способствует повышению энергоэффективности, справедливости распределения и надёжности распределённых вычислительных систем.
\end{itemize}

\textbf{Обоснованность и достоверность} полученных результатов обеспечивается:
\begin{enumerate}
	\item Математической строгостью разработанных моделей и алгоритмов, основанных на признанных теориях (теория игр, аукционная теория, MARL);
	\item Комплексной верификацией компонентов на синтетических данных с известным ожидаемым результатом;
	\item Валидацией на разнообразных симуляционных сценариях, включая стресс-тесты и анализ чувствительности;
	\item Независимостью экспериментов и воспроизводимостью результатов через использование фиксированных random seeds и детальной документацией параметров.
\end{enumerate}

\textbf{Апробация работы.} Результаты, изложенные в данной диссертации, докладывались и обсуждались на конференциях:
\begin{enumerate}
	\item 11-я Международная конференция <<Распределенные вычисления и грид-технологии в науке и образовании>> (GRID'2025), Российская Федерация, г. Дубна, 07.07.2025–11.07.2025;
	\item TODO
\end{enumerate}

\textbf{Публикации.} TODO Основные результаты по теме диссертации изложены в ... печатных изданиях, из которых ... — в периодических научных журналах, индексируемых Scopus [...], ... — в периодических научных журналах, индексируемых РИНЦ [...]. Получены ... свидетельства о государственной регистрации программ для ЭВМ [...]. 

\textbf{Объем и структура работы.} Диссертация состоит из~введения,
\formbytotal{totalchapter}{глав}{ы}{}{},
заключения и
\formbytotal{totalappendix}{приложен}{ия}{ий}{}.
%% на случай ошибок оставляю исходный кусок на месте, закомментированным
%Полный объём диссертации составляет  \ref*{TotPages}~страницу
%с~\totalfigures{}~рисунками и~\totaltables{}~таблицами. Список литературы
%содержит \total{citenum}~наименований.
%
Полный объём диссертации составляет
\formbytotal{TotPages}{страниц}{у}{ы}{}, включая
\formbytotal{totalcount@figure}{рисун}{ок}{ка}{ков} и
\formbytotal{totalcount@table}{таблиц}{у}{ы}{}.
Список литературы содержит
\formbytotal{citenum}{наименован}{ие}{ия}{ий}.

\textbf{Во введении} сформулированы критерии, показана актуальность и
новизна исследования, описана теоретическая и практическая значимость полученных результатов, обозначена цель и задачи исследования для ее достижения,
сформулированы положения, выносимые на защиту.

\textbf{В первой главе} дан подробный анализ существующих подходов к распределению ресурсов в IoT, рассмотрены ключевые технологии (аукционы, MARL, Dec-POMDP), выявлены пробелы и мотивация исследования.

\textbf{Во второй главе} представлены два основных алгоритма: MA-VCG (мультиагентный VCG) и Dec-POMDP-MARL (децентрализованный POMDP с QMIX). Описаны архитектура системы, математические модели, формальные определения и методология.

\textbf{В третьей главе} приведены результаты обширных экспериментов на различных масштабах сети, анализ масштабируемости, устойчивости к отказам и адаптивности. Исследовано влияние параметров среды (задержки, дедлайны, распределение нагрузки).

\textbf{В четвертой главе} рассмотрена конкретная предметная область -- управление ресурсами в автоматизированных логистических системах. Описаны сценарии, результаты моделирования и практические рекомендации.

\textbf{В заключении} подведены итоги исследования, обсуждены ограничения, определены направления дальнейшей работы.

\textbf{Основные положения, выносимые на защиту:}
\begin{enumerate}
	\item Интеграция VCG-аукционов с многоагентным обучением обеспечивает эффективное, справедливое и адаптивное управление распределёнными вычислительными ресурсами в условиях неполной информации и динамических изменений среды.
	\item Применение алгоритма QMIX в контексте Dec-POMDP позволяет достичь линейной масштабируемости управления при сохранении гарантий конвергенции и качества решения.
	\item Архитектура с централизованным обучением и децентрализованным исполнением обеспечивает гибкость в адаптации параметров аукциона и гарантирует доминантно-стратегическую совместимость механизма.
	\item Предложенный подход применим к реальным сценариям управления ресурсами в роботизированных складах и демонстрирует существенное улучшение показателей эффективности по сравнению с базовыми методами.
\end{enumerate}

 % Характеристика работы по структуре во введении и в автореферате не отличается (ГОСТ Р 7.0.11, пункты 5.3.1 и 9.2.1), потому её загружаем из одного и того же внешнего файла, предварительно задав форму выделения некоторым параметрам